\begin{abstract}[it]
Il presente lavoro di tesi riguarda la progettazione e implementazione di un sistema di \acf{BI} e di un Data Lake per l'azienda UNOX S.p.A, sfruttando le tecnologie messe a disposizione dalla piattaforma \acf{AWS}. Il progetto si colloca in un contesto aziendale in cui l'analisi rapida e precisa delle informazioni è cruciale per migliorare le performance operative e supportare decisioni strategiche basate sui dati.

Il sistema di BI realizzato consente di raccogliere e analizzare dati provenienti da fonti diverse, contribuendo a ridurre le inefficienze, segnalare eventuali criticità, individuare nuovi flussi di ricavi e identificare aree di crescita futura. Attraverso la creazione di un Data Lake centralizzato e l’automazione dei processi di integrazione e analisi dei dati, si è reso possibile ottimizzare la gestione delle informazioni aziendali, garantendo una visione complessiva e dettagliata delle prestazioni.

Il sistema è stato progettato per essere facilmente accessibile anche da team aziendali, tra cui Ricerca e Sviluppo, IT e altri reparti tecnici, grazie all'uso di strumenti intuitivi come AWS QuickSight, che consentono un'interazione semplice e visiva con i dati, senza necessità di conoscenze specifiche in DBMS e linguaggi di query. Questo approccio ha permesso di rendere i dati fruibili a tutti i livelli aziendali, favorendo una maggiore autonomia nella loro analisi.

L'implementazione ha utilizzato componenti AWS quali Amazon S3, Glue e QuickSight, garantendo una gestione scalabile, sicura e completamente automatizzata dei dati. Il risultato finale è un sistema integrato di Business Intelligence che riduce significativamente i tempi di analisi e reporting, migliorando la capacità decisionale e promuovendo una cultura aziendale data-driven.
\end{abstract}